% Options for packages loaded elsewhere
\PassOptionsToPackage{unicode}{hyperref}
\PassOptionsToPackage{hyphens}{url}
%
\documentclass[
  11pt,
  ignorenonframetext,
]{beamer}
\usepackage{pgfpages}
\setbeamertemplate{caption}[numbered]
\setbeamertemplate{caption label separator}{: }
\setbeamercolor{caption name}{fg=normal text.fg}
\beamertemplatenavigationsymbolsempty
% Prevent slide breaks in the middle of a paragraph
\widowpenalties 1 10000
\raggedbottom
\setbeamertemplate{part page}{
  \centering
  \begin{beamercolorbox}[sep=16pt,center]{part title}
    \usebeamerfont{part title}\insertpart\par
  \end{beamercolorbox}
}
\setbeamertemplate{section page}{
  \centering
  \begin{beamercolorbox}[sep=12pt,center]{part title}
    \usebeamerfont{section title}\insertsection\par
  \end{beamercolorbox}
}
\setbeamertemplate{subsection page}{
  \centering
  \begin{beamercolorbox}[sep=8pt,center]{part title}
    \usebeamerfont{subsection title}\insertsubsection\par
  \end{beamercolorbox}
}
\AtBeginPart{
  \frame{\partpage}
}
\AtBeginSection{
  \ifbibliography
  \else
    \frame{\sectionpage}
  \fi
}
\AtBeginSubsection{
  \frame{\subsectionpage}
}
\usepackage{amsmath,amssymb}
\usepackage{setspace}
\usepackage{iftex}
\ifPDFTeX
  \usepackage[T1]{fontenc}
  \usepackage[utf8]{inputenc}
  \usepackage{textcomp} % provide euro and other symbols
\else % if luatex or xetex
  \usepackage{unicode-math} % this also loads fontspec
  \defaultfontfeatures{Scale=MatchLowercase}
  \defaultfontfeatures[\rmfamily]{Ligatures=TeX,Scale=1}
\fi
\usepackage{lmodern}
\usetheme[]{Berlin}
\usecolortheme{dove}
\usefonttheme{structurebold}
\ifPDFTeX\else
  % xetex/luatex font selection
\fi
% Use upquote if available, for straight quotes in verbatim environments
\IfFileExists{upquote.sty}{\usepackage{upquote}}{}
\IfFileExists{microtype.sty}{% use microtype if available
  \usepackage[]{microtype}
  \UseMicrotypeSet[protrusion]{basicmath} % disable protrusion for tt fonts
}{}
\makeatletter
\@ifundefined{KOMAClassName}{% if non-KOMA class
  \IfFileExists{parskip.sty}{%
    \usepackage{parskip}
  }{% else
    \setlength{\parindent}{0pt}
    \setlength{\parskip}{6pt plus 2pt minus 1pt}}
}{% if KOMA class
  \KOMAoptions{parskip=half}}
\makeatother
\usepackage{xcolor}
\newif\ifbibliography
\usepackage{color}
\usepackage{fancyvrb}
\newcommand{\VerbBar}{|}
\newcommand{\VERB}{\Verb[commandchars=\\\{\}]}
\DefineVerbatimEnvironment{Highlighting}{Verbatim}{commandchars=\\\{\}}
% Add ',fontsize=\small' for more characters per line
\usepackage{framed}
\definecolor{shadecolor}{RGB}{248,248,248}
\newenvironment{Shaded}{\begin{snugshade}}{\end{snugshade}}
\newcommand{\AlertTok}[1]{\textcolor[rgb]{0.94,0.16,0.16}{#1}}
\newcommand{\AnnotationTok}[1]{\textcolor[rgb]{0.56,0.35,0.01}{\textbf{\textit{#1}}}}
\newcommand{\AttributeTok}[1]{\textcolor[rgb]{0.13,0.29,0.53}{#1}}
\newcommand{\BaseNTok}[1]{\textcolor[rgb]{0.00,0.00,0.81}{#1}}
\newcommand{\BuiltInTok}[1]{#1}
\newcommand{\CharTok}[1]{\textcolor[rgb]{0.31,0.60,0.02}{#1}}
\newcommand{\CommentTok}[1]{\textcolor[rgb]{0.56,0.35,0.01}{\textit{#1}}}
\newcommand{\CommentVarTok}[1]{\textcolor[rgb]{0.56,0.35,0.01}{\textbf{\textit{#1}}}}
\newcommand{\ConstantTok}[1]{\textcolor[rgb]{0.56,0.35,0.01}{#1}}
\newcommand{\ControlFlowTok}[1]{\textcolor[rgb]{0.13,0.29,0.53}{\textbf{#1}}}
\newcommand{\DataTypeTok}[1]{\textcolor[rgb]{0.13,0.29,0.53}{#1}}
\newcommand{\DecValTok}[1]{\textcolor[rgb]{0.00,0.00,0.81}{#1}}
\newcommand{\DocumentationTok}[1]{\textcolor[rgb]{0.56,0.35,0.01}{\textbf{\textit{#1}}}}
\newcommand{\ErrorTok}[1]{\textcolor[rgb]{0.64,0.00,0.00}{\textbf{#1}}}
\newcommand{\ExtensionTok}[1]{#1}
\newcommand{\FloatTok}[1]{\textcolor[rgb]{0.00,0.00,0.81}{#1}}
\newcommand{\FunctionTok}[1]{\textcolor[rgb]{0.13,0.29,0.53}{\textbf{#1}}}
\newcommand{\ImportTok}[1]{#1}
\newcommand{\InformationTok}[1]{\textcolor[rgb]{0.56,0.35,0.01}{\textbf{\textit{#1}}}}
\newcommand{\KeywordTok}[1]{\textcolor[rgb]{0.13,0.29,0.53}{\textbf{#1}}}
\newcommand{\NormalTok}[1]{#1}
\newcommand{\OperatorTok}[1]{\textcolor[rgb]{0.81,0.36,0.00}{\textbf{#1}}}
\newcommand{\OtherTok}[1]{\textcolor[rgb]{0.56,0.35,0.01}{#1}}
\newcommand{\PreprocessorTok}[1]{\textcolor[rgb]{0.56,0.35,0.01}{\textit{#1}}}
\newcommand{\RegionMarkerTok}[1]{#1}
\newcommand{\SpecialCharTok}[1]{\textcolor[rgb]{0.81,0.36,0.00}{\textbf{#1}}}
\newcommand{\SpecialStringTok}[1]{\textcolor[rgb]{0.31,0.60,0.02}{#1}}
\newcommand{\StringTok}[1]{\textcolor[rgb]{0.31,0.60,0.02}{#1}}
\newcommand{\VariableTok}[1]{\textcolor[rgb]{0.00,0.00,0.00}{#1}}
\newcommand{\VerbatimStringTok}[1]{\textcolor[rgb]{0.31,0.60,0.02}{#1}}
\newcommand{\WarningTok}[1]{\textcolor[rgb]{0.56,0.35,0.01}{\textbf{\textit{#1}}}}
\setlength{\emergencystretch}{3em} % prevent overfull lines
\providecommand{\tightlist}{%
  \setlength{\itemsep}{0pt}\setlength{\parskip}{0pt}}
\setcounter{secnumdepth}{5}
\usepackage{booktabs}
\usepackage{longtable}
\usepackage{array}
\usepackage{multirow}
\usepackage{wrapfig}
\usepackage{float}
\usepackage{colortbl}
\usepackage{pdflscape}
\usepackage{tabu}
\usepackage{threeparttable}
\usepackage{threeparttablex}
\usepackage[normalem]{ulem}
\usepackage{makecell}
\usepackage{xcolor}
\usepackage{tabularray}
\usepackage[normalem]{ulem}
\usepackage{graphicx}
\usepackage{rotating}
\UseTblrLibrary{booktabs}
\UseTblrLibrary{siunitx}
\NewTableCommand{\tinytableDefineColor}[3]{\definecolor{#1}{#2}{#3}}
\newcommand{\tinytableTabularrayUnderline}[1]{\underline{#1}}
\newcommand{\tinytableTabularrayStrikeout}[1]{\sout{#1}}
\ifLuaTeX
  \usepackage{selnolig}  % disable illegal ligatures
\fi
\usepackage{bookmark}
\IfFileExists{xurl.sty}{\usepackage{xurl}}{} % add URL line breaks if available
\urlstyle{same}
\hypersetup{
  pdftitle={WHO Health Data Analysis},
  pdfauthor={Marjella Ernst},
  hidelinks,
  pdfcreator={LaTeX via pandoc}}

\title{WHO Health Data Analysis}
\subtitle{A Presentation by Marjella Ernst}
\author{Marjella Ernst}
\date{June 2025}

\begin{document}
\frame{\titlepage}

\setstretch{1.2}
\begin{frame}
\end{frame}

\section{\texorpdfstring{Introduction
\label{Introduction}}{Introduction }}\label{introduction}

\begin{frame}{Introduction \label{Introduction}}
This presentation explores the relationship of various factors such as:
- stress - physical exercise - sleep, and

on health.
\end{frame}

\section{Hypotheses to test}\label{hypotheses-to-test}

\begin{frame}{Hypotheses to test}
\begin{enumerate}
\tightlist
\item
  sleeping is more important to health than exercise
\item
  living a stress free lifestyle has a major impact on health
\end{enumerate}
\end{frame}

\section{Research Question}\label{research-question}

\begin{frame}{Research Question}
``Which are the main influencing factors on health and how can we use
these insights to improve it?''
\end{frame}

\section{Data}\label{data}

\begin{frame}{Data}
\begin{itemize}
\tightlist
\item
  The data used for this analysis originates from the \textbf{World
  Health Organization (WHO)}.
\item
  It includes self-reported indicators on:

  \begin{itemize}
  \tightlist
  \item
    Physical activity
  \item
    Sleep quality
  \item
    Stress levels
  \item
    Caloric intake
  \end{itemize}
\item
  The dataset builds the empirical basis for testing the relationship
  between lifestyle factors and health outcomes
\end{itemize}
\end{frame}

\section{Methodology}\label{methodology}

\begin{frame}[fragile]{Methodology}
\begin{itemize}
\item
  A multiple linear regression model is used to estimate the effect of:

  \begin{itemize}
  \tightlist
  \item
    \textbf{Exercise} (physical activity),
  \item
    \textbf{Sleep quality}, and
  \item
    \textbf{Stress level}
  \end{itemize}

  on the dependent variable \textbf{Health} (approximated by excess
  caloric intake).
\item
  The model is specified as: \[
  \text{Health}_i = \beta_0 + \beta_1 \cdot \text{Exercise}_i + \beta_2 \cdot \text{Sleep}_i + \beta_3 \cdot \text{Stress}_i + \varepsilon_i
  \]
\item
  In R, the model is estimated as:
\end{itemize}

\begin{Shaded}
\begin{Highlighting}[]
\FunctionTok{lm}\NormalTok{(Health }\SpecialCharTok{\textasciitilde{}}\NormalTok{ Exercise }\SpecialCharTok{+}\NormalTok{ Sleep }\SpecialCharTok{+}\NormalTok{ Stress, }\AttributeTok{data =}\NormalTok{ processed)}
\end{Highlighting}
\end{Shaded}
\end{frame}

\section{Challenge: What is Health?}\label{challenge-what-is-health}

\begin{frame}{Challenge: What is Health?}
\begin{itemize}
\tightlist
\item
  For the purpose of this analysis, health is approximated by excess
  body weight

  \begin{itemize}
  \tightlist
  \item
    This serves as a proxy indicator due to data limitations
  \item
    Other common indicators, such as BMI, are not available in the
    dataset
  \end{itemize}
\end{itemize}
\end{frame}

\section{Regression Results}\label{regression-results}

\begin{frame}{Regression Results}
\begin{minipage}{\linewidth}
\scriptsize
\begin{table}
\centering
\begin{talltblr}[         %% tabularray outer open
caption={Regression Results: Health \textasciitilde{} Exercise + Sleep + Stress},
note{}={+ p \num{< 0.1}, * p \num{< 0.05}, ** p \num{< 0.01}, *** p \num{< 0.001}},
]                     %% tabularray outer close
{                     %% tabularray inner open
colspec={Q[]Q[]},
column{2}={}{halign=c,},
column{1}={}{halign=l,},
hline{10}={1,2}{solid, black, 0.05em},
}                     %% tabularray inner close
\toprule
& (1) \\ \midrule %% TinyTableHeader
(Intercept) & \num{246.148}** \\
& (\num{87.845}) \\
Exercise & \num{285.774}*** \\
& (\num{21.394}) \\
Sleep & \num{30.220} \\
& (\num{20.625}) \\
Stress & \num{-5.357} \\
& (\num{8.851}) \\
Num.Obs. & \num{100} \\
R2 & \num{0.652} \\
R2 Adj. & \num{0.642} \\
AIC & \num{1370.7} \\
BIC & \num{1383.7} \\
Log.Lik. & \num{-680.331} \\
RMSE & \num{217.97} \\
\bottomrule
\end{talltblr}
\end{table}
\end{minipage}
\end{frame}

\begin{frame}{Interpretation of Results I}
\phantomsection\label{interpretation-of-results-i}
\begin{itemize}
\tightlist
\item
  \textbf{Exercise} has strong and highly significantly positive effect
  on health

  \begin{itemize}
  \tightlist
  \item
    Coefficient: 285.77 (***), p \textless{} 0.001\\
  \item
    Interpretation: Higher physical activity is strongly associated with
    better health (less excess weight)
  \end{itemize}
\item
  \textbf{Sleep} shows small positive effect, but is not statistically
  significant

  \begin{itemize}
  \tightlist
  \item
    Coefficient: 30.22, p \textgreater{} 0.1\\
  \item
    No strong evidence that sleep quality alone explains variation in
    health
  \end{itemize}
\end{itemize}
\end{frame}

\begin{frame}{Interpretation of Results II}
\phantomsection\label{interpretation-of-results-ii}
\begin{itemize}
\tightlist
\item
  \textbf{Stress} has a small negative effect, but is also not
  statistically significant

  \begin{itemize}
  \tightlist
  \item
    Coefficient: --5.36, p \textgreater{} 0.5\\
  \item
    Cannot conclude that stress significantly affects health in this
    model
  \end{itemize}
\item
  \textbf{Model fit}:

  \begin{itemize}
  \tightlist
  \item
    R² = 0.652 → model explains \textasciitilde65\% of the variation in
    health
  \item
    Adjusted R² = 0.642 is still high after adjusting for model
    complexity
  \item
    AIC = 1370.7 model complexity vs.~fit trade-off
  \end{itemize}
\end{itemize}
\end{frame}

\section{Limitations}\label{limitations}

\begin{frame}{Limitations}
\begin{itemize}
\tightlist
\item
  Health is difficult to measure because it includes multiple
  dimensions:

  \begin{itemize}
  \tightlist
  \item
    Physical aspects (e.g., number of illnesses, fitness)
  \item
    Mental aspects (e.g., stress resistance, absence of psychological
    conditions)
  \end{itemize}
\item
  Many influencing factors of health are themselves difficult to observe
  or quantify:

  \begin{itemize}
  \tightlist
  \item
    Genetics, sleep, nutrition
  \item
    Environmental and external factors
  \item
    Access to health care
  \end{itemize}
\item
  There is a significant risk of reverse causality:

  \begin{itemize}
  \tightlist
  \item
    Poor health can increase stress
  \item
    Illness may limit exercise
  \item
    Health problems can disrupt sleep
  \end{itemize}
\end{itemize}
\end{frame}

\section{Conclusion}\label{conclusion}

\begin{frame}{Conclusion}
\begin{itemize}
\item
  Exercise is found to be clearly statistically significant for health
\item
  Sleep and stress show expected effects but lack statistical
  significance
\item
  The model explains a large share of variation in health
  (\textasciitilde65\%)
\item
  Hower, health is a multifaceted concept, more complex than this model
  can explain
\item
  Further research should include:

  \begin{itemize}
  \tightlist
  \item
    Better proxies for health (e.g., BMI, blood pressure, clinical
    diagnoses)
  \item
    Longitudinal data to address reverse causality
  \item
    Interaction effects (like stress and sleep)
  \end{itemize}
\end{itemize}
\end{frame}

\end{document}
